\documentclass[10pt, a4paper, onecolumn, openany]{book} % openany make chapter start whenever, DELETE in OFFICIAL

% PACKAGES:
% Font Encoding
\usepackage[utf8]{inputenc}     % Use UTF-8
\usepackage[T1]{fontenc}        % T1 font encoding (latin characters)
% Header
\usepackage{fancyhdr}           % fancy page header options
\usepackage{titlesec}           % used to have \thechapter in same line as \chaptertitlename
% MISC
\usepackage{hyperref}           % \url{}
\usepackage{xurl}
\usepackage{graphicx}           % images
\usepackage{xcolor}            % colors
\usepackage{fancyvrb}          % colors in Verbatin, header: \begin{Verbatim}[commandchars=\\\{\}]

% DECORATIVE LINES + CHAPTER IN SAME LINE:
\renewcommand{\headrulewidth}{2pt}  % Top decorative line
\renewcommand{\footrulewidth}{2pt}  % Bottom decorative line
\pagestyle{fancy}                   % better header for normal pages, not only chapter ones
\fancyhf{}                          % clear header and adjust as wanted:
    \chead{\leftmark}               % header
    \cfoot{Page \thepage}           % footer
\fancypagestyle{plain}{
\fancyhf{} 
    \chead{\leftmark}       % header
    \cfoot{Page \thepage}   % footer
}
\renewcommand{\chaptername}{}       % change word chapter to {}
\titleformat{\chapter}[hang]{\normalfont\huge\bfseries}{\chaptertitlename\ \thechapter.}{1em}{} % Chapter in same line as chapter name

% SIZES OF SECTIONS:
\titleformat*{\section}{\LARGE\bfseries}
\titleformat*{\subsection}{\Large\bfseries}
\titleformat*{\subsubsection}{\large\bfseries}

% DISABLE huge space after (paragraph indent) section name before text starts:
\setlength{\parindent}{0pt}

% COLOR EXAMPLES:
% \definecolor{MyColor}{RGB}{219, 48, 122}  % define
% \textcolor{MyColor}{Some random text}     % usage in document
\definecolor{root}{RGB}{222, 0, 0}
\definecolor{user}{RGB}{0, 150, 00}
\definecolor{dir}{RGB}{0, 100, 200}
\definecolor{file}{RGB}{77, 187, 101}
\definecolor{block}{RGB}{255, 80, 0}
\definecolor{command}{RGB}{41, 182, 0}
\definecolor{comment}{RGB}{0, 182, 182}

% COLORS FOR CODE in document:
%\begin{minted}[frame=lines,framesep=2mm,baselinestretch=1.2,fontsize=\footnotesize,linenos]{js}
%\end{minted}

% IMAGES:
\graphicspath{./images/} % define directory
% \includegraphics[scale=1.5]{./images/random_image.png} % usage in document

% TABLE:
% \begin{center}
%    \begin{small}
%    \begin{tabular}{|p{1cm}|p{1cm}|p{1cm}|p{1cm}|p{1cm}|p{1cm}|p{1cm}|}
%    \hline
%    + & - & * & / & \% & ** & () \\
%    \hline
%    1 & 2 & 3 & 4 & 5 & 6 & 7 \\
%    \hline
%    \end{tabular}
%    \end{small}
%\end{center}


%\titlespacing*{\section}{0pt}{1.5cm}{0.2cm}
%\titlespacing*{\subsection}{0pt}{0.2cm}{0.2cm}

%%%%%%%%%%%%%%%%%%%%%%%%%%%%%%%%%%%%%%%%%%%%%%%%%%%%%%%%%%%%%%%%%%%%%%%%%%%%
%%%%%%%%%%%%%%%%%%%%%%%%%%%%%%%%% TITLE %%%%%%%%%%%%%%%%%%%%%%%%%%%%%%%%%%%%
%%%%%%%%%%%%%%%%%%%%%%%%%%%%%%%%%%%%%%%%%%%%%%%%%%%%%%%%%%%%%%%%%%%%%%%%%%%%
\title{\textbf{Raspberry Pi 4}}
\author{AISK}
\date{June, 2021}
%%%%%%%%%%%%%%%%%%%%%%%%%%%%%%%%%%%%%%%%%%%%%%%%%%%%%%%%%%%%%%%%%%%%%%%%%%%%
%%%%%%%%%%%%%%%%%%%%%%%%%%%%%%%%% START %%%%%%%%%%%%%%%%%%%%%%%%%%%%%%%%%%%%
%%%%%%%%%%%%%%%%%%%%%%%%%%%%%%%%%%%%%%%%%%%%%%%%%%%%%%%%%%%%%%%%%%%%%%%%%%%%
\begin{document}
\maketitle
%\clearpage % official blank page
\tableofcontents

%%%%%%%%%%%%%%%%%%%%%%%%%%%%%%%%%%%%%%%%%%%%%%%%%%%%%%%%%%%%%%%%%%%%%%%%%%%%
%%%%%%%%%%%%%%%%%%%%%%%%%%%%%%%%% Hardware %%%%%%%%%%%%%%%%%%%%%%%%%%%%%%%%%%%%
%%%%%%%%%%%%%%%%%%%%%%%%%%%%%%%%%%%%%%%%%%%%%%%%%%%%%%%%%%%%%%%%%%%%%%%%%%%%
\chapter{Hardware}
\section{Hardware requirements}
\subsection{Raspberry Pi 4 Model B (2,4 or 8GB)}
\begin{center}
    \includegraphics[scale=0.3]{./images/raspberry_pi_4_model_b.png}
\end{center}
\begin{center}
    \begin{small}
    \begin{tabular}{|p{4cm}|p{4cm}|}
    \hline
    \textbf{Model} & \textbf{Price (EUR)} \\
    \hline
    2 GB & \ 53.90 \\
    \hline
    4 GB & \ 77.90 \\
    \hline
    8 GB & 104.90 \\
    \hline
    \end{tabular}
    \end{small}
\end{center}

\subsection{Raspberry Pi 15.3W USB-C Power Supply}
\begin{center}
    \includegraphics[scale=0.1]{./images/raspberry_pi_4_model_b_psu.png}
\end{center}    
\begin{center}
    \begin{small}
    \begin{tabular}{|p{4cm}|p{4cm}|}
    \hline
    \textbf{Parameter} & \textbf{Value} \\
    \hline
    Input & AC 100-240V 50/60Hz  \\
    \hline
    Output & \ DC 5.1V 3A \\
    \hline
    Maximum Output Power & 15.3W \\
    \hline
    Color & Black/White \\
    \hline
    Price (EUR) & 13.90 \\
    \hline
    \end{tabular}
    \end{small}
\end{center}

\subsection{microSD card (SanDisk micro SDHC 32 GB Extreme Pro A1 UHS-I (V30) + SD adapter)}
\begin{center}
    \includegraphics[scale=0.3]{./images/SanDisk_micro_SDHC_32_GB.png}
\end{center}
\begin{center}
    \begin{small}
    \begin{tabular}{|p{4cm}|p{4cm}|}
    \hline
    \textbf{Parameter} & \textbf{Value} \\
    \hline
    Recommended Size & > 32 GB \\
    \hline
    Filesystem & \ VFAT (exFAT) \\
    \hline
    Diagnostic test & Passed \\
    \hline
    Price (EUR) & 14.90 \\
    \hline
    \end{tabular}
    \end{small}
\end{center}

\subsection{Cable - HDMI (M) to HDMI micro (M) (OEM RASPBERRY Pi HDMI 3m)}
\begin{center}
    \includegraphics[scale=0.2]{./images/oem_raspberry_pi_hdmi.png}
\end{center}
\begin{center}
    \begin{small}
    \begin{tabular}{|p{4cm}|p{4cm}|}
    \hline
    \textbf{Parameter} & \textbf{Value} \\
    \hline
    Port 1 & HDMI (M) \\
    \hline
    Port 2 & \ HDMI micro) \\
    \hline
    Length & 3m \\
    \hline
    Price (EUR) & 7.00 \\
    \hline
    \end{tabular}
    \end{small}
\end{center}

\section{Optional Hardware}
\subsection{RASPBERRY Pi 4 case}
\begin{center}
    \includegraphics[scale=0.2]{./images/raspberry_pi_4_case_black.png}
\end{center}
\begin{center}
    \begin{small}
    \begin{tabular}{|p{4cm}|p{4cm}|}
    \hline
    \textbf{Parameter} & \textbf{Value} \\
    \hline
    Color & Black/White \\
    \hline
    Price (EUR) & 7.50 \\
    \hline
    \end{tabular}
    \end{small}
\end{center}

\subsection{Raspberry Pi 4 Case Fan}
\begin{center}
    \includegraphics[scale=0.2]{./images/raspberry_pi_4_case_fan.png}
\end{center}
\begin{center}
    \begin{small}
    \begin{tabular}{|p{4cm}|p{4cm}|}
    \hline
    \textbf{Parameter} & \textbf{Value} \\
    \hline
    Price (EUR) & 5.30 \\
    \hline
    \end{tabular}
    \end{small}
\end{center}

\subsection{USB-C cable with circuit break for Raspberry Pi 4B, 27cm}
\begin{center}
    \includegraphics[scale=0.15]{./images/usb_c_cable_raspberry-pi_4b_27cm.png}
\end{center}
\begin{center}
    \begin{small}
    \begin{tabular}{|p{4cm}|p{4cm}|}
    \hline
    \textbf{Parameter} & \textbf{Value} \\
    \hline
    Price (EUR) & 7.42 \\
    \hline
    \end{tabular}
    \end{small}
\end{center}

\section{Other Documentation}
\subsection{Hardware documentation}
\url{https://www.raspberrypi.org/documentation/hardware/raspberrypi/bcm2711/rpi_DATA_2711_1p0_preliminary.pdf}


%%%%%%%%%%%%%%%%%%%%%%%%%%%%%%%%%%%%%%%%%%%%%%%%%%%%%%%%%%%%%%%%%%%%%%%%%%%%
%%%%%%%%%%%%%%%%%%%%%%%%%%%%%%%%% OS Installation %%%%%%%%%%%%%%%%%%%%%%%%%%%%%%%%%%%%
%%%%%%%%%%%%%%%%%%%%%%%%%%%%%%%%%%%%%%%%%%%%%%%%%%%%%%%%%%%%%%%%%%%%%%%%%%%%
\chapter{OS Installation}
\section{Download OS}
\underline{\url{https://www.raspberrypi.org/software/operating-systems/}}\newline
\underline{\url{https://downloads.raspberrypi.org/raspios_lite_arm64/images/}}
\subsection{Check SHA256 checksum}
\begin{Verbatim}[commandchars=\\\{\}]
    \textcolor{user}{pi\$} \textcolor{command}{sha256sum} <YYYY-MM-dd-raspios-buster-armhf.zip>
\end{Verbatim}
\section{Format SD card}
\subsection{SD card partitioning}
\begin{Verbatim}[commandchars=\\\{\}]
    \textcolor{root}{pi#} \textcolor{command}{parted} \textcolor{block}{/dev/sdX}
    \textcolor{user}{(parted)} mktable gpt
    \textcolor{user}{(parted)} unit mib
    \textcolor{user}{(parted)} mkpart primary 1 -1
    \textcolor{user}{(parted)} q
\end{Verbatim}

\subsection{VFAT < 32GB (NOT NEEDED)}
% if problems -> GPARTED -> msdos -> fat32
\begin{Verbatim}[commandchars=\\\{\}]
Check for partition:
    \textcolor{user}{pi\$} \textcolor{command}{lsblk} -pf
\textit{Umount FS at first in case is automounted!}

Create VFAT filesystem:
    \textcolor{root}{pi#} \textcolor{command}{mkfs} -t vfat <\textcolor{block}{/dev/sdXY}>
\end{Verbatim}
\subsection{exFAT > 32GB (NOT NEEDED)}
\begin{Verbatim}[commandchars=\\\{\}]
Format:
    \textcolor{root}{pi#} \textcolor{command}{mkfs.exfat} <\textcolor{block}{/dev/sdXY}>
    
Check:
    \textcolor{root}{pi#} \textcolor{command}{fsck.exfat} <\textcolor{block}{/dev/sdXY}>
\end{Verbatim}

\section{Unzip and Flash OS}
\begin{Verbatim}[commandchars=\\\{\}]
    \textcolor{user}{pi\$} \textcolor{command}{unzip} <YYYY-MM-dd-raspios-buster-armhf.zip>

    \textcolor{root}{pi#} \textcolor{command}{dd} if=</PATH/TOYYYY-MM-dd-raspios-buster-armhf.img>    
    of=<\textcolor{block}{/dev/sdX}> bs=1M conv=fsync [status=progress]
    
Note: wait a few minutes till FS loads first, if necessary, restart monitor.
\end{Verbatim}

\section{WiP - First boot monitorless access:}
\begin{Verbatim}[commandchars=\\\{\}]
mount /dev/sdX1 /mnt/<dir>/

\textcolor{dir}{/mnt/<dir>/}\textcolor{file}{wpa_supplicant.conf}:
ctrl_interface=DIR=/var/run/wpa_supplicant GROUP=netdev
update_config=1
country=<2 LETTER ISO CODE>

network=\{
    ssid="<ESSID>"
    scan_ssid=1 \textcolor{comment}{# Find hidden network}
    key_mgmt=WPA-PSK WPA-EAP
    \textcolor{comment}{#psk="<PLAINTEXT-PASSWD>"}
    psk=<32byte-HEX-NUMBER>
    priority=1 \textcolor{comment}{# To which WiFi connect first}
\}

umount /dev/sdX1
\end{Verbatim}

\section{Finishing}
\begin{Verbatim}[commandchars=\\\{\}]
Remove SD card and place it back to Raspberry Pi as in picture:
\end{Verbatim}
\begin{center}
    \includegraphics[scale=0.2]{./images/sd_card_to_pi.png}
\end{center}
\begin{Verbatim}[commandchars=\\\{\}]
Connect HDMI, keyboard and mouse.
Plug in USB-C power cable.
\end{Verbatim}

\subsection{Default Credentials}
\begin{Verbatim}[commandchars=\\\{\}]
Username: \textbf{pi}
Password: \textbf{raspberry}
\end{Verbatim}

\subsection{Update}
\begin{Verbatim}[commandchars=\\\{\}]
    \textcolor{root}{pi#} \textcolor{command}{apt} update
    \textcolor{root}{pi#} \textcolor{command}{apt} <upgrade|full-upgrade>
\end{Verbatim}


%%%%%%%%%%%%%%%%%%%%%%%%%%%%%%%%%%%%%%%%%%%%%%%%%%%%%%%%%%%%%%%%%%%%%%%%%%%%
%%%%%%%%%%%%%%%%%%%%%%%%%%%%%%%%% Firs Start %%%%%%%%%%%%%%%%%%%%%%%%%%%%%%%%%%%%
%%%%%%%%%%%%%%%%%%%%%%%%%%%%%%%%%%%%%%%%%%%%%%%%%%%%%%%%%%%%%%%%%%%%%%%%%%%%
\chapter{First Start}
\section{Display}
\begin{Verbatim}[commandchars=\\\{\}]
- Remove black borders (\textcolor{file}{/boot/config.txt}):
disable_overscan=1
\end{Verbatim}

\section{GPU RAM}
\begin{Verbatim}[commandchars=\\\{\}]
(\textcolor{file}{/boot/config.txt}):
gpu_mem=512
\end{Verbatim}
\section{Rfkill}
\begin{Verbatim}[commandchars=\\\{\}]
- Block BT, unblock WiFi:
    \textcolor{user}{pi\$} \textcolor{command}{rfkill} list
    0: phy0: Wireless LAN
        Soft blocked: yes
        Hard blocked: no
    1: hci0: Bluetooth
        Soft blocked: no
        Hard blocked: no
    
    \textcolor{user}{pi\$} \textcolor{command}{rfkill} block bluetooth
    \textcolor{user}{pi\$} \textcolor{command}{rfkill} unblock wlan

    \textcolor{user}{pi\$} \textcolor{command}{rfkill} list
    0: phy0: Wireless LAN
        Soft blocked: no
        Hard blocked: no
    1: hci0: Bluetooth
        Soft blocked: yes
        Hard blocked: no
\end{Verbatim}

\section{Password}
\begin{Verbatim}[commandchars=\\\{\}]
- User password change:
    \textcolor{user}{pi\$} \textcolor{command}{passwd}
   
- Root password set/change: 
    \textcolor{root}{pi#} \textcolor{command}{passwd} root
\end{Verbatim}

\section{Timezone}
\begin{Verbatim}[commandchars=\\\{\}]
- Show current timezone:
    \textcolor{user}{pi\$} \textcolor{command}{timedatectl} [-a]

- List available timezones:
    \textcolor{user}{pi\$} \textcolor{command}{timedatectl} list-timezones
    
- Change timezone:
    \textcolor{root}{pi#} \textcolor{command}{timedatectl} set-timezone <UTC|Europe/Copenhagen>
\end{Verbatim}

\section{Locale}
\begin{Verbatim}[commandchars=\\\{\}]
- Show current locales:
    \textcolor{user}{pi\$} \textcolor{command}{locale}

- Change locale (\textcolor{file}{/etc/default/locale}):
    LANG=en_US.UTF-8
    \textcolor{comment}{# First day in a week MON, not SUN:}
    LC_TIME="en_GB.UTF-8"
    \textcolor{comment}{# Default paper size:}
    LC_PAPER="en_GB.UTF-8"
    LC_MEASUREMENT="en_GB.UTF-8"
\end{Verbatim}

\section{Keyboard Layout}
\begin{Verbatim}[commandchars=\\\{\}]
Available keyboards \textcolor{dir}{/usr/share/keymaps/i386/}

- Set keyboard (\textcolor{file}{/etc/default/keyboard}):
XKBMODEL="pc105"
XKBLAYOUT="us"
XKBVARIANT=""
XKBOPTIONS=""
BACKSPACE="guess"
\end{Verbatim}

\section{Hostname}
\begin{Verbatim}[commandchars=\\\{\}]
1. Display hostname:
    \textcolor{user}{pi\$} \textcolor{command}{hostname}
    
2. Change hostname:
File (\textcolor{file}{/etc/hostname}):
<HOSTNAME>

File (\textcolor{file}{/etc/hosts}):
...
127.0.1.1   <HOSTNAME>
\end{Verbatim}


%%%%%%%%%%%%%%%%%%%%%%%%%%%%%%%%%%%%%%%%%%%%%%%%%%%%%%%%%%%%%%%%%%%%%%%%%%%%
%%%%%%%%%%%%%%%%%%%%%%%%%%%%%%%%% Network %%%%%%%%%%%%%%%%%%%%%%%%%%%%%%%%%%%%
%%%%%%%%%%%%%%%%%%%%%%%%%%%%%%%%%%%%%%%%%%%%%%%%%%%%%%%%%%%%%%%%%%%%%%%%%%%%
\chapter{Network}
\section{Scheme}
\begin{Verbatim}[commandchars=\\\{\}]
                          +--------+                               +---------+
WAN oooo dhcpcd:68 (DHCP) |aidkrp01| (192.168.0.1) dnsmasq:67 ---- |L2 switch|
                          +--------+                               +---------+
\end{Verbatim}

\section{Disable unnecessary}
\subsection{Uninstall Bluetooth}
\begin{Verbatim}[commandchars=\\\{\}]
List packages:
    \textcolor{user}{pi} \textcolor{command}{dpkg} -l | \textcolor{command}{grep} -i bluetooth

Purge packages:
    \textcolor{root}{pi#} \textcolor{command}{dpkg} --purge <bluez bluez-firmware pi-bluetooth>
\end{Verbatim}
\subsection{Disable IPv6}
\subsubsection{Disable in system}
\begin{Verbatim}[commandchars=\\\{\}]
File: (\textcolor{file}{/etc/sysctl.conf}):
net.ipv6.conf.all.disable_ipv6 = 1

Temporary file: \textcolor{file}{/proc/sys/net/ipv6/conf/all/disable_ipv6}
\end{Verbatim}
\subsubsection{Disable in FW}
\begin{Verbatim}[commandchars=\\\{\}]
- Install persistent iptables:
    \textcolor{root}{pi#} \textcolor{command}{apt} install iptables-persistent

- DROP all IPv6 traffic:
    \textcolor{root}{pi#} \textcolor{command}{ip6tables} -F
    \textcolor{root}{pi#} \textcolor{command}{ip6tables} -P INPUT DROP
    \textcolor{root}{pi#} \textcolor{command}{ip6tables} -P FORWARD DROP
    \textcolor{root}{pi#} \textcolor{command}{ip6tables} -P OUTPUT DROP 

- Update rules:
    \textcolor{root}{pi#} \textcolor{command}{ip6tables-save} > \textcolor{file}{/etc/iptables/rules.v6}
\end{Verbatim}


\subsection{Disable Avahi daemon}
\begin{Verbatim}[commandchars=\\\{\}]
    \textcolor{root}{pi#} \textcolor{command}{systemctl} disable avahi-daemon.socket
    \textcolor{root}{pi#} \textcolor{command}{systemctl} disable avahi-daemon.service
\end{Verbatim}

\section{WLAN}
\subsection{Connect automatically to AP}
\begin{Verbatim}[commandchars=\\\{\}]
File: (\textcolor{file}{/etc/network/interfaces}):
\textcolor{comment}{# bring interface UP when is detected by kernel:}
#auto wlan0 # start on boot time
allow-hotplug wlan0
\textcolor{comment}{# use DHCP:}
iface wlan0 inet dhcp
\textcolor{comment}{# use wpa_supplicant conf file:}
wpa-conf /etc/wpa_supplicant/wpa_supplicant.conf

- Disable unnecessary service (error caused by "wpa-conf ..."):
    \textcolor{root}{pi#} \textcolor{command}{systemctl} disable networking.service
\end{Verbatim}
\subsection{WiFi Settings}
\begin{Verbatim}[commandchars=\\\{\}]
File: (\textcolor{file}{/etc/wpa_supplicant/wpa_supplicant.conf}) (0600):
\textcolor{comment}{# Basic settings and language for zones:}
ctrl_interface=DIR=/var/run/wpa_supplicant GROUP=netdev
update_config=1
country=<2 LETTER ISO CODE>

\textcolor{comment}{# Password protected:}
network=\{
    ssid="<ESSID>"
    scan_ssid=1 \textcolor{comment}{# Find hidden network}
    key_mgmt=WPA-PSK WPA-EAP
    \textcolor{comment}{#psk="<PLAINTEXT-PASSWD>"}
    psk=<32byte-HEX-NUMBER>
    priority=1 \textcolor{comment}{# To which WiFi connect first}
\}

\textcolor{comment}{# Unprotected:}
network=\{
    ssid="<ESSID>"
    scan_ssid=1 \textcolor{comment}{# Find hidden network}
    key_mgmt=NONE
    priority=2 \textcolor{comment}{# To which WiFi connect first}
\}
\end{Verbatim}

\subsection{DHCP client}
\begin{Verbatim}[commandchars=\\\{\}]
- Install dhcpcd:
    \textcolor{root}{root#} \textcolor{command}{apt} install dhcpcd5

- Purge old DHCP:
    \textcolor{root}{root#} \textcolor{command}{dpkg} --purge <isc-dhcp-client isc-dhcp-common>
    
- Configure DHCP:
File: (\textcolor{file}{/etc/dhcpcd.conf}):
\textcolor{comment}{# Inform DHCP server of our hostname for DDNS.}
###hostname
\textcolor{comment}{# Use the hardware address of the interface for the Client ID.}
###clientid
\textcolor{comment}{# Persist interface configuration when dhcpcd exits.}
###persistent
\textcolor{comment}{# A list of options to request from the DHCP server.}
###option domain_name_servers, domain_name, domain_search, host_name
\end{Verbatim}
\subsection{DNS}
\subsubsection{Configure}
\begin{Verbatim}[commandchars=\\\{\}]
File: (\textcolor{file}{/etc/resolv.conf}) (0644):
\textcolor{comment}{# Uncensored DNS - Denmark - Unicast}
nameserver 89.233.43.71
#nameserver 1.1.1.1
#nameserver 1.0.0.1
\textcolor{comment}{# CZ.NIC}
nameserver 193.17.47.1
nameserver 185.43.135.1
\end{Verbatim}

\subsection{Change MAC address on startup}
\begin{Verbatim}[commandchars=\\\{\}]
Change MAC address (XN:XX:XX:XX:XX:XX), N = 0,2,4,6,8,A,C,E:
(\underline{\url{https://macvendors.com/}})

File (\textcolor{file}{/etc/systemd/system/<changemac>@.service}):
[Unit]
Description=changes mac for \%I
Wants=network.target
Before=network.target
BindsTo=sys-subsystem-net-devices-\%i.device
After=sys-subsystem-net-devices-\%i.device

[Service]
Type=oneshot
ExecStart=/usr/sbin/ip l set address a8:96:8a:6e:b2:58 \%I
RemainAfterExit=yes

[Install]
WantedBy=multi-user.target

- Enable service on startup:
    \textcolor{root}{pi#} \textcolor{command}{systemctl} enable changemac@<wlan0>.service
\end{Verbatim}

\section{Ethernet}
\subsection{Interface config}
\begin{Verbatim}[commandchars=\\\{\}]
File: (\textcolor{file}{/etc/network/interfaces}):
\textcolor{comment}{# bring interface UP when is detected by kernel:}
#auto eth0 # start on boot time
allow-hotplug eth0
\textcolor{comment}{# use static IP:}
iface eth0 inet static
    address <192.168.0.1>
    netmask <255.255.255.0>
    #gateway <192.168.0.1>
    #dns-nameservers <1.1.1.1>
    #dns-nameservers <1.0.0.1>
\end{Verbatim}

\subsection{DHCP server}
\subsubsection{Installation}
\begin{Verbatim}[commandchars=\\\{\}]
    \textcolor{root}{pi#} \textcolor{command}{apt} install dnsmasq
    
- Example config file:
(\underline{\url{https://github.com/imp/dnsmasq/blob/master/dnsmasq.conf.example}})
\end{Verbatim}

\subsubsection{Configuration}
\begin{Verbatim}[commandchars=\\\{\}]
File (\textcolor{file}{/etc/dnsmasq.conf}):
\textcolor{comment}{# DNS SETTINGS:}
\textcolor{comment}{# DNS port (default 53), 0 - disable DNS:}
port=0
\textcolor{comment}{# Which DNS server should be used for IP resolving:}
#server=1.1.1.1
#server=1.0.0.1
\textcolor{comment}{# DHCP SETTINGS:}
\textcolor{comment}{# Specify interface:}
#interface=<wlan0>
\textcolor{comment}{# DHCP range,lease-time:}
dhcp-range=eth0,192.168.0.10,192.168.0.254,255.255.255.0,12h
#dhcp-range=[eth0],192.168.0.10,192.168.0.254,[255.255.255.0],[12h]
\textcolor{comment}{Set the limit on DHCP leases, the default is 150}
#dhcp-lease-max=150
\textcolor{comment}{# Static IP address (MAC,IPv4):}
#dhcp-host=XX:XX:XX:XX:XX:XX,10.0.1.2
\textcolor{comment}{# Blacklist MAC:}
#dhcp-host=XX:XX:XX:XX:XX:XX,ignore
\textcolor{comment}{# Specify NTP server:}
#dhcp-option=option:ntp-server,X.X.X.X[,X.X.X.X]
\textcolor{comment}{# Change default TTL to 50:}
#dhcp-option=23,50
\end{Verbatim}

\subsubsection{Enable DHCP server}
\begin{Verbatim}[commandchars=\\\{\}]
    \textcolor{root}{pi#} \textcolor{command}{systemctl} enable dnsmasq.service
\end{Verbatim}

\subsection{NAT}
\subsubsection{IPv4 forwarding}
\begin{Verbatim}[commandchars=\\\{\}]
File: (\textcolor{file}{/etc/sysctl.conf}):
net.ipv4.conf.eth0.forwarding=1
net.ipv4.conf.wlan0.forwarding=1

Temporary file: \textcolor{file}{/proc/sys/net/ipv4/conf/<eth0|wlan0>/forwarding}
\end{Verbatim}

\subsubsection{NAT iptables rules}
\begin{Verbatim}[commandchars=\\\{\}]
- Install persistent iptables:
    \textcolor{root}{pi#} \textcolor{command}{apt} install iptables-persistent

- Set NAT forwarding:
    \textcolor{root}{pi#} \textcolor{command}{iptables} -F
    \textcolor{root}{pi#} \textcolor{command}{iptables} -t nat -A POSTROUTING -o <wlan0> -j MASQUERADE
    \textcolor{root}{pi#} \textcolor{command}{iptables} -A FORWARD -m conntrack --ctstate RELATED,ESTABLISHED
    -j ACCEPT
    \textcolor{root}{pi#} \textcolor{command}{iptables} -A FORWARD -i <eth0> -o <wlan0> -j ACCEPT

- Update rules:
    \textcolor{root}{pi#} \textcolor{command}{iptables-save} > \textcolor{file}{/etc/iptables/rules.v4}
\end{Verbatim}

\subsubsection{WiP - ToDo}
\begin{Verbatim}[commandchars=\\\{\}]
WHAT IS NOT WORKING:
- not getting any DNS server from DHCP server
- do not respond to icmp
- proxy blocking via /etc/hosts not working
- bridge with hidden AP

FW + ips for wlan0
ssh setup 
vnc setup
logging
honeypot
\end{Verbatim}


%%%%%%%%%%%%%%%%%%%%%%%%%%%%%%%%%%%%%%%%%%%%%%%%%%%%%%%%%%%%%%%%%%%%%%%%%%%%
%%%%%%%%%%%%%%%%%%%%%%%%%%%%%%%%% Shell and Texteditor %%%%%%%%%%%%%%%%%%%%%%%%%%%%%%%%%%%%
%%%%%%%%%%%%%%%%%%%%%%%%%%%%%%%%%%%%%%%%%%%%%%%%%%%%%%%%%%%%%%%%%%%%%%%%%%%%
\chapter{Shell and Texteditor}
\section{vim}
\subsection{Installation}
\begin{Verbatim}[commandchars=\\\{\}]
    \textcolor{root}{pi#} \textcolor{command}{apt} install vim
\end{Verbatim}
\subsection{Configure}
\begin{Verbatim}[commandchars=\\\{\}]
Config files:
(\textcolor{file}{/etc/vim/vimrc}) - global
(\textcolor{file}{~/.vimrc}) - per user
\underline{\url{https://github.com/AISK11/raspberrypi4/blob/main/dotfiles/.vimrc}}
\end{Verbatim}
\subsection{Set as default}
\begin{Verbatim}[commandchars=\\\{\}]
    \textcolor{root}{pi#} \textcolor{command}{update-alternatives} --config editor
\end{Verbatim}
\subsection{Tricks}
\begin{Verbatim}[commandchars=\\\{\}]
- Add '#' to start of file:
A)  :\%s/^/#/g
B)  \textcolor{root}{pi#} \textcolor{command}{vim} -c ":\%s/^/#/g" -c ":wq" <FILE>
\end{Verbatim}

\section{zsh}
\subsection{Install and setup}
\begin{Verbatim}[commandchars=\\\{\}]
Install:
    \textcolor{root}{pi#} \textcolor{command}{apt} install zsh zsh-autosuggestions zsh-syntax-highlighting
    
Set up zsh as default \$\{SHELL\}:
a)  \textcolor{user}{pi\$} \textcolor{command}{chsh} -s \textcolor{file}{/bin/zsh}
b)  \textcolor{root}{pi#} \textcolor{command}{usermod} -s \textcolor{file}{/bin/zsh} <USER>
\end{Verbatim}
\subsection{Configure}
\begin{Verbatim}[commandchars=\\\{\}]
File (\textcolor{file}{~/.zshrc}) (0644):
\underline{\url{https://github.com/AISK11/raspberrypi4/blob/main/dotfiles/.zshrc}}
\end{Verbatim}


%%%%%%%%%%%%%%%%%%%%%%%%%%%%%%%%%%%%%%%%%%%%%%%%%%%%%%%%%%%%%%%%%%%%%%%%%%%%
%%%%%%%%%%%%%%%%%%%%%%%%%%%%%%%%% X + i3 %%%%%%%%%%%%%%%%%%%%%%%%%%%%%%%%%%%%
%%%%%%%%%%%%%%%%%%%%%%%%%%%%%%%%%%%%%%%%%%%%%%%%%%%%%%%%%%%%%%%%%%%%%%%%%%%%
\chapter{X + i3}
\section{X}
\subsection{Installation}
\begin{Verbatim}[commandchars=\\\{\}]
    \textcolor{root}{pi#} \textcolor{command}{apt} --no-install-recommends install xserver-xorg
    xserver-xorg-video-fbdev xinit x11-xserver-utils
\end{Verbatim}
\subsection{Start X on tty1}
\begin{Verbatim}[commandchars=\\\{\}]
File (\textcolor{file}{~/.bash_profile}||\textcolor{file}{~/.zshrc}) (0644):
\textcolor{comment}{# if (length of \$DISPLAY is 0 && \$(tty) = /dev/tty1) startx}
if [[ -z \$DISPLAY ]] && [[ \$(tty) = /dev/tty1 ]]; 
then
    source /etc/profile
    startx
fi
\end{Verbatim}


\section{i3}
\subsection{i3 Installation (do not install with i3-gaps)}
\begin{Verbatim}[commandchars=\\\{\}]
    \textcolor{root}{pi#} \textcolor{command}{apt} install i3 --no-install-recommends
    \textcolor{root}{pi#} \textcolor{command}{apt} install i3-wm 
\end{Verbatim}
\subsection{i3-gaps from source}
\begin{Verbatim}[commandchars=\\\{\}]
URL: (\underline{https://github.com/Airblader/i3/wiki/Building-from-source})

- Clone i3-gaps:
    \textcolor{user}{pi\$} \textcolor{command}{cd} \textcolor{dir}{/etc/}
    \textcolor{root}{pi#} \textcolor{command}{git} clone https://www.github.com/Airblader/i3 i3-gaps
    \textcolor{user}{pi\$} \textcolor{command}{cd} \textcolor{dir}{./i3-gaps/}


- Install dependencies
    \textcolor{root}{pi#} \textcolor{command}{apt} install meson
    \textcolor{root}{pi#} \textcolor{command}{apt} install dh-autoreconf libxcb-keysyms1-dev 
libxcb-util0-dev xcb libxcb1-dev libxcb-icccm4-dev libyajl-dev
libev-dev libxcb-xkb-dev libxcb-cursor-dev libxkbcommon-dev 
libxcb-xinerama0-dev libxkbcommon-x11-dev libpango1.0-dev
libstartup-notification0-dev libxcb-randr0-dev libxcb-xrm0 
libxcb-xrm-dev libxcb-shape0 libxcb-shape0-dev

- Compile:
    \textcolor{root}{pi#} \textcolor{command}{mkdir} -p \textcolor{dir}{build} && cd \textcolor{dir}{build}
    \textcolor{root}{pi#} \textcolor{command}{meson} --prefix \textcolor{file}{/usr/local}
    \textcolor{root}{pi#} \textcolor{command}{ninja}
    \textcolor{root}{pi#} \textcolor{command}{ninja} install
\end{Verbatim}

\subsection{Run i3 after Xorg starts}
\begin{Verbatim}[commandchars=\\\{\}]
File (\textcolor{file}{~/.xinitrc}) (0644):
\underline{\url{https://github.com/AISK11/raspberrypi4/blob/main/dotfiles/.xinitrc}}
\end{Verbatim}

\subsection{i3config}
\subsubsection{Installation}
\begin{Verbatim}[commandchars=\\\{\}]
    \textcolor{root}{pi#} \textcolor{command}{apt} install i3blocks i3lock numlockx rofi feh scrot
    compton
\end{Verbatim}
\subsubsection{Configuration}
\begin{Verbatim}[commandchars=\\\{\}]   
File (\textcolor{file}{~/.config/i3/config}) (0644):
\underline{\url{https://github.com/AISK11/raspberrypi4/blob/main/dotfiles/.config/i3/config}}
\end{Verbatim}

\section{urxvt}
\subsection{Installation}
\begin{Verbatim}[commandchars=\\\{\}]
    \textcolor{root}{pi#} \textcolor{command}{apt} install rxvt-unicode-256color
\end{Verbatim}

\subsection{Configuration}
\begin{Verbatim}[commandchars=\\\{\}]
File (\textcolor{file}{~/.Xresources}) (0644):
\underline{\url{https://github.com/AISK11/raspberrypi4/blob/main/dotfiles/.Xresources}}

- Refresh X database for file and restart urxvt:
    \textcolor{root}{pi#} \textcolor{command}{xrdb} \textcolor{file}{~/.Xresources}

- Execute read in X database every time i3 is started:
File (\textcolor{file}{~/.config/i3/config}) (0644):
exec xrdb ~/.Xresources    
\end{Verbatim}

\subsection{Set as default emulator}
\begin{Verbatim}[commandchars=\\\{\}]    
    \textcolor{root}{pi#} \textcolor{command}{update-alternatives} --config x-terminal-emulator
\end{Verbatim}

\section{i3blocks}
\begin{Verbatim}[commandchars=\\\{\}]
i3blocks visual:
File (\textcolor{file}{/etc/i3blocks.conf}) (0644):
\underline{\url{https://github.com/AISK11/raspberrypi4/blob/main/config_files/i3blocks.conf}}

Required scripts:
Dir (\textcolor{dir}{~/.config/i3/scripts/}):
\underline{\url{https://github.com/AISK11/raspberrypi4/tree/main/dotfiles/.config/i3/scripts}}
\end{Verbatim}

\section{Fonts}
\subsection{View Font}
\begin{Verbatim}[commandchars=\\\{\}]
- Install:
    \textcolor{root}{pi} \textcolor{command}{apt} imagemagick
    
- Use:
    \textcolor{user}{pi\$} \textcolor{command}{fc-list} 
    \textcolor{user}{pi\$} \textcolor{command}{display} </PATH/TO/font.ttf>
\end{Verbatim}
\subsection{font-awesome}
\begin{Verbatim}[commandchars=\\\{\}]
    \textcolor{user}{pi\$} \textcolor{command}{mkdir} \textcolor{dir}{~/.fonts/}
    \textcolor{user}{pi\$} \textcolor{command}{git clone} https://github.com/FortAwesome/Font-Awesome
    \textcolor{user}{pi\$} \textcolor{command}{cd} \textcolor{dir}{Font-Awesome}
    \textcolor{user}{pi\$} \textcolor{command}{find} \textcolor{dir}{.} -regex ".*\char92.ttf\$" -o -regex ".*\char92.otf\$" -exec cp \{\} \textcolor{dir}{~/.fonts/} \char92;
    \textcolor{user}{pi\$} \textcolor{command}{rm} -rf \textcolor{dir}{~/Font-Awesome/}
\end{Verbatim}

\section{lxappearance}
\subsection{lxappearance}
\begin{Verbatim}[commandchars=\\\{\}]
- Install:
    \textcolor{root}{pi} \textcolor{command}{apt} install lxappearance
    
- Usage;
    \textcolor{user}{pi\$} \textcolor{command}{lxappearance}
\end{Verbatim}
\subsection{GTK Theme + Icons}
\begin{Verbatim}[commandchars=\\\{\}]
Directories:
    \textcolor{dir}{.themes}
    \textcolor{dir}{.icons}
\end{Verbatim}
\subsection{Cursor}
\begin{Verbatim}[commandchars=\\\{\}]
URL: \underline{https://github.com/varlesh/volantes-cursors}

- Install dependencies:
    \textcolor{root}{pi#} \textcolor{command}{apt} install git make inkscape xcursorgen

- Install cursor:
    \textcolor{user}{pi\$} \textcolor{command}{git} clone https://github.com/varlesh/volantes-cursors
    \textcolor{user}{pi\$} \textcolor{command}{cd} \textcolor{dir}{volantes-cursors}
    \textcolor{user}{pi\$} \textcolor{command}{make} build
    \textcolor{root}{pi#} \textcolor{command}{make} install
\end{Verbatim}


%%%%%%%%%%%%%%%%%%%%%%%%%%%%%%%%%%%%%%%%
%%%%%%%%%%%%%%%%%%%%%%%%%%%%%%%%%%%%%%%%%%%%%%%%%%%%%%%%%%%%%%%%%%%%%%%%%%%%
%%%%%%%%%%%%%%%%%%%%%%%%%%%%%%%%% ? %%%%%%%%%%%%%%%%%%%%%%%%%%%%%%%%%%%%
%%%%%%%%%%%%%%%%%%%%%%%%%%%%%%%%%%%%%%%%%%%%%%%%%%%%%%%%%%%%%%%%%%%%%%%%%%%%
\chapter{Chia blockchain}
\section{Documentation}
\begin{Verbatim}[commandchars=\\\{\}]
Make sure to use 64bit version (Raspbian 64).
\end{Verbatim}
\underline{\url{https://github.com/Chia-Network/chia-blockchain/wiki/Raspberry-Pi}}

\section{Increase SWAP to 1GB}
\begin{Verbatim}[commandchars=\\\{\}]
- Turn off swap:
    \textcolor{root}{pi#} \textcolor{command}{dphys-swapfile} swapoff
    
- Modify swap file (\textcolor{file}{/etc/dphys-swapfile}):
\textcolor{comment}{# at least 1024 is needed:}
CONF_SWAPSIZE=2048

- Delete original swap file and recreate new swap file:
    \textcolor{root}{pi#} \textcolor{command}{dphys-swapfile} setup
     
- Turn on swap:
    \textcolor{root}{pi#} \textcolor{command}{dphys-swapfile} swapon
\end{Verbatim}


\section{Setup}
\begin{Verbatim}[commandchars=\\\{\}]
- Download dependencies:
    \textcolor{root}{pi#} \textcolor{command}{apt} install -y build-essential python3-dev
    
- Check if 64bit adchitecture of python is installed:
    \textcolor{user}{pi#} \textcolor{command}{python3} -c 'import platform; print(platform.architecture))'
OUTPUT: ('64bit', 'ELF')
\end{Verbatim}

\section{Install chia-blockchain}
\subsection{CLI installation}
\begin{Verbatim}[commandchars=\\\{\}]
    \textcolor{user}{pi#} \textcolor{command}{git} clone https://github.com/Chia-Network/chia-blockchain.git
    -b latest
    \textcolor{user}{pi#} \textcolor{command}{cd} \textcolor{dir}{chia-blockchain/}
    \textcolor{user}{pi#} \textcolor{command}{sh} \textcolor{file}{install.sh}
    \textcolor{user}{pi#} \textcolor{dir}{.} \textcolor{file}{./activate}
    \textcolor{user}{(venv) pi#} \textcolor{command}{chia} init
\end{Verbatim}
\subsection{Generate/Import keys}
\begin{Verbatim}[commandchars=\\\{\}]
A) Generate:
    \textcolor{user}{(venv) pi#} \textcolor{command}{chia} keys generate
B) Import:
    \textcolor{user}{(venv) pi#} \textcolor{command}{chia} keys add
\end{Verbatim}
\subsection{GUI installation}
\begin{Verbatim}[commandchars=\\\{\}]
    \textcolor{user}{(venv) pi#} \textcolor{command}{sh} install-gui.sh
    \textcolor{user}{(venv) pi#} \textcolor{command}{cd} \textcolor{dir}{chia-blockchain-gui/}
    
- Fix error with not found shared object "libnss3.so":
    \textcolor{user}{pi#} \textcolor{command}{apt} \textcolor{command}{install} libnss3 libnss3-dev
    
- Fix error with "OSError; [Errno 99] error while attempting to bind 
on address ('::1', 55400, 0, 0): cannot assign requested address. 
closing code: 1":
(\textcolor{file}{~/chia-blockchain/chia/util/initial-config.yaml}):
#self_hostname: &self_hostname "localhost"
self_hostname: &self_hostname "127.0.0.1"
(\textcolor{file}{/etc/hosts}):
#::1    localhost ip6-localhost ip6-loopback

- Start GUI:    
    \textcolor{user}{(venv) pi#} \textcolor{command}{npm} run electron &
\end{Verbatim}

\section{Harddrives}
\subsection{Mount HDDs}
\begin{Verbatim}[commandchars=\\\{\}]
- Create directories:
    \textcolor{root}{pi#} \textcolor{command}{mkdir} -p \textcolor{dir}{/mnt/chia/hdd01}
    \textcolor{root}{pi#} \textcolor{command}{mkdir} -p \textcolor{dir}{/mnt/chia/hdd02}

- Mount drives:
    \textcolor{root}{pi#} \textcolor{command}{mount} <\textcolor{block}{/dev/sda1}> \textcolor{dir}{/mnt/chia/hdd01}
    \textcolor{root}{pi#} \textcolor{command}{mount} <\textcolor{block}{/dev/sdb1}> \textcolor{dir}{/mnt/chia/hdd01}

- Configure fstab:
1. see UUIDs of block devices:
    \textcolor{root}{pi#} \textcolor{command}{blkid}

2. configure fstab:
File (\textcolor{file}{/etc/fstab}) (0644):
\textcolor{comment}{# <fs>          <mountpoint>        <type>  <opts>      <dump/pass>}
\textcolor{comment}{# UUID of /dev/sdXY:}
UUID="<UUID>"   /mnt/chia/hdd01     ext4    defaults,noatime    0 2
UUID="<UUID>"   /mnt/chia/hdd02     ext4    defaults,noatime    0 2


/etc/fstab ????

crontab for harddrives

crontab for auto update

\end{Verbatim}





\chapter{Packages and Services}
\section{VRMS - Find non free software}
\subsection{Installation}
\begin{Verbatim}[commandchars=\\\{\}]
    \textcolor{root}{pi#} \textcolor{command}{apt} install vrms
\end{Verbatim}
\subsection{Run}
\begin{Verbatim}[commandchars=\\\{\}]
    \textcolor{user}{pi\$} \textcolor{command}{vrms}
\end{Verbatim}

%\noindent\makebox[\linewidth]{\rule{\paperwidth}{0.4pt}}

\section{List Packages And Processes}
\subsection{Packages}
\begin{Verbatim}[commandchars=\\\{\}]
All packages:
    \textcolor{user}{pi\$} \textcolor{command}{dpkg} -l
    
List files from package:
    \textcolor{user}{pi\$} \textcolor{command}{dpkg} -L <PACKAGE>
    
Find package to whom file belongs:
    \textcolor{user}{pi\$} \textcolor{command}{dpkg} -S <FILE>
\end{Verbatim}

\subsection{Processes}
\begin{Verbatim}[commandchars=\\\{\}]
- List all processes:
    \textcolor{user}{pi\$} \textcolor{command}{ps} -eF
    
- List everything [or only services]:
    \textcolor{user}{pi\$} \textcolor{command}{systemctl} --all [--type service]
\end{Verbatim}



\section{Disable unnecessary software}
\subsection{Bluetooth}
\subsubsection{Disable on boot}
\begin{Verbatim}[commandchars=\\\{\}]
File (\textcolor{file}{/boot/config.txt}) (0644):
\textcolor{comment}{# Disable Bluetooth:}
dtoverlay=disable-bt
\end{Verbatim}
\subsubsection{Purge Bluetooth}
\begin{Verbatim}[commandchars=\\\{\}]
List packages:
    \textcolor{user}{pi\$} \textcolor{command}{dpkg} -l | \textcolor{command}{grep} -i bluetooth

Purge packages:
    \textcolor{root}{pi#} \textcolor{command}{dpkg} --purge <bluez bluez-firmware pi-bluetooth>
\end{Verbatim}
\subsection{CUPS}
\subsubsection{Purge CUPS}
\begin{Verbatim}[commandchars=\\\{\}]
List packages:
    \textcolor{user}{pi\$} \textcolor{command}{dpkg} -l | \textcolor{command}{grep} -i cups

Purge packages:
    \textcolor{root}{pi#} \textcolor{command}{dpkg} --purge <cups cups-browsed cups-browsed cups-client 
    cups-common cups-core-drivers cups-daemon 
    cups-filters cups-filters-core-drivers cups-ipp-utils 
    cups-ppdc cups-server-common>
\end{Verbatim}

\subsection{Avahi}
\subsubsection{Disable on boot}
\begin{Verbatim}[commandchars=\\\{\}]
    \textcolor{root}{pi#} \textcolor{command}{systemctl} disable avahi-daemon.socket 
    \textcolor{root}{pi#} \textcolor{command}{systemctl} disable avahi-daemon.service
\end{Verbatim}

\subsection{Other}
\begin{Verbatim}[commandchars=\\\{\}]
alsa-utils:
    \textcolor{root}{pi#} \textcolor{command}{systemctl} disable alsa-utils 
\end{Verbatim}

\subsection{Uninstall unused packages}
\begin{Verbatim}[commandchars=\\\{\}]
    \textcolor{root}{pi#} \textcolor{command}{apt} autoremove 
\end{Verbatim}




\chapter{Security}\section{No MOTD}
\begin{Verbatim}[commandchars=\\\{\}]
    \textcolor{root}{pi#} \textcolor{command}{echo} "" > \textcolor{file}{/etc/issue}
\end{Verbatim}

\section{Fork Bomb}
\subsection{Execution}
\begin{Verbatim}[commandchars=\\\{\}]
    \textcolor{user}{pi\$} \textcolor{command}{:()\{ :|:& \};:}
\end{Verbatim}
\subsection{Mitigation}
\begin{Verbatim}[commandchars=\\\{\}]
For current session:
    \textcolor{user}{pi\$#} \textcolor{command}{ulimit} -H -u 5000

Permanently - file \textcolor{file}{/etc/security/limits.conf} (0644):
<USER> hard nproc 5000
\end{Verbatim}
\subsection{Mitigation in real time}
\begin{Verbatim}[commandchars=\\\{\}]
- man 7 signal for process sigs
    \textcolor{root}{pi#} \textcolor{command}{killall} -SIGSTOP -u <USER>
    \textcolor{root}{pi#} \textcolor{command}{killall} -SIGKILL -u <USER>
\end{Verbatim}

\section{FW}
\subsection{List rules}
\begin{Verbatim}[commandchars=\\\{\}]
    \textcolor{root}{pi#} \textcolor{command}{iptables}  -L -n -v
\end{Verbatim}

\subsection{Default INPUT Policy}
\begin{Verbatim}[commandchars=\\\{\}]
    \textcolor{root}{pi#} \textcolor{command}{iptables}  -P <\underline{INPUT}|OUTPUT> <\underline{ACCEPT}|DROP>
\end{Verbatim}

\subsection{Filter incoming traffic}
\begin{center}
    \begin{small}
    \begin{tabular}{|p{4cm}|p{4cm}|}
    \hline
    \textbf{Function} & \textbf{Port Range} \\
    \hline
    well-known ports & 0 - 1023  \\
    \hline
    registered ports & 1024 - 49151 \\
    \hline
    private ports & 49152 - 65535 \\
    \hline
    \end{tabular}
    \end{small}
\end{center}
\begin{Verbatim}[commandchars=\\\{\}]
DROP - no response; REJECT - closed port.

Drop ports from any subnet on TCP ports 0-21 and 23-1023:
    \textcolor{root}{pi#} \textcolor{command}{iptables} -A INPUT -s 0.0.0.0/0 -p tcp --match multiport
    --dports 0:21,23:1023 -j DROP
    
- Log traffic (\textcolor{file}{/var/log/syslog}):    
    \textcolor{root}{pi#} \textcolor{command}{iptables} -N LOG_AND_DROP
    \textcolor{root}{pi#} \textcolor{command}{iptables} -A LOG_AND_DROP -j LOG --log-prefix "iptables denied: "
    --log-level 7
    \textcolor{root}{pi#} \textcolor{command}{iptables} -A LOG_AND_DROP -j DROP    
    \textcolor{root}{pi#} \textcolor{command}{iptables} -A INPUT -s 0.0.0.0/0 -p tcp --match multiport
    --dports 0:21,23:1023 -j LOG_AND_DROP
\end{Verbatim}

\subsection{Flush all settings}
\begin{Verbatim}[commandchars=\\\{\}]
    \textcolor{root}{pi#} \textcolor{command}{iptables} -F
\end{Verbatim}

\subsection{Block all IPv6 traffic}
\begin{Verbatim}[commandchars=\\\{\}]
List IPv6 traffic:
    \textcolor{root}{pi#} \textcolor{command}{ip6tables} -L -n -v   

Drop all policies:
    \textcolor{root}{pi#} \textcolor{command}{ip6tables} -P INPUT DROP
    \textcolor{root}{pi#} \textcolor{command}{ip6tables} -P FORWARD DROP
    \textcolor{root}{pi#} \textcolor{command}{ip6tables} -P OUTPUT DROP 
\end{Verbatim}

\subsection{Persistent iptables}
\begin{Verbatim}[commandchars=\\\{\}]
Installation:
    \textcolor{root}{pi#} \textcolor{command}{apt} install iptables-persistent

Update rules:
    \textcolor{root}{root#} \textcolor{command}{iptables-save} > \textcolor{file}{/etc/iptables/rules.v4}
    \textcolor{root}{root#} \textcolor{command}{ip6tables-save} > \textcolor{file}{/etc/iptables/rules.v6}
\end{Verbatim}

\section{Proxy Filter}
\begin{Verbatim}[commandchars=\\\{\}]
- File: (\textcolor{file}{/etc/hosts}):
0.0.0.0 www.malicious.com
\end{Verbatim}

\section{Logs}
\subsection{Login}
\begin{Verbatim}[commandchars=\\\{\}]
- see last successfull logins (\textcolor{file}{/var/log/wtmp}):
    \textcolor{user}{pi\$} \textcolor{command}{last} [-aiF]

- see last unsuccessfull logins (\textcolor{file}{/var/log/btmp}):
    \textcolor{root}{pi#} \textcolor{command}{lastb} [-aiF]
    
- see sudo logs:
    \textcolor{user}{pi\$} \textcolor{command}{grep} "sudo" \textcolor{file}{/var/log/auth.log}
\end{Verbatim}

\subsection{?}
\begin{Verbatim}[commandchars=\\\{\}]
/etc/rsyslog.conf
    
What logs are where
Log rotation (logrotate)


man logrotate
\url{https://www.linode.com/docs/guides/use-logrotate-to-manage-log-files/}
\end{Verbatim}

\section{IDS and IPS}
\begin{Verbatim}[commandchars=\\\{\}]
snort, fail2ban, ossec
\end{Verbatim}

\section{Monitoring}
\subsection{ntpd}
\begin{Verbatim}[commandchars=\\\{\}]
- Installation:
    \textcolor{root}{pi#} apt install ntp

- Enable:
    \textcolor{root}{pi#} systemctl enable ntp.service
\end{Verbatim}

\subsection{Zabbix}
\begin{Verbatim}[commandchars=\\\{\}]
- Original documentation:
\url{https://www.zabbix.com/download?zabbix=5.4&os_distribution=raspberry_pi_os&os_version=10_buster&db=mysql&ws=apache}

- Installation:
    \textcolor{root}{pi#} wget https://repo.zabbix.com/zabbix/5.4/raspbian/pool/main/z/zabbix-release/zabbix-release_5.4-1+debian10_all.deb
    \textcolor{root}{pi#} dpkg -i zabbix-release_5.4-1+debian10_all.deb
    \textcolor{root}{pi#} apt update 
    \textcolor{root}{pi#} apt install zabbix-server-mysql zabbix-frontend-php 
    zabbix-apache-conf zabbix-sql-scripts zabbix-agent 

- Create database:
    \textcolor{root}{pi#} mysql -uroot -p
password: *****
    mysql> create database zabbix character set utf8 collate utf8_bin;
    mysql> create user zabbix@localhost identified by '<PASSWORD>';
    mysql> grant all privileges on zabbix.* to zabbix@localhost;
    mysql> quit;
    zcat /usr/share/doc/zabbix-sql-scripts/mysql/create.sql.gz | mysql -uzabbix -p zabbix

- Configure database:
File (\textcolor{file}{/etc/zabbix/zabbix_server.conf}):
DBName=zabbix
DBUser=zabbix
DBPassword=<PASSWORD> 

- Start and enable zabbix processes:
    \textcolor{root}{pi#} systemctl restart zabbix-server.service zabbix-agent.service apache2.service 
    \textcolor{root}{pi#} systemctl enable zabbix-server zabbix-agent apache2 

- BUG fix (not showing MySQL):
    \textcolor{root}{pi#} apt install php7.3-mysql

- Connect to zabbix frontend:
http://<HOSTNAME|IP>/zabbix
 
- Frontend config:
Database type: MySQL / PostgreSQL
Database host: localhost
Database port: 0 (default = 3306)
Database name: zabbix
User: zabbix
Password: *****
Host: localhost
Port: 10051
Name: zabbix (frontend page name only, not important)
Default time zone: System (UTC)
Default theme; Blue

- FW ports:
TCP: 80, 443, 10050, 10051

- Default credentials:
Admin
zabbix



/etc/zabbix/web/zabbix.conf.php



new tty/ssh login
new ssh established connection
low disk space
CPU/RAM/TEMPERATURE load for 5 minutes
IDS/IPS logs
no internet connection
turn on/off
honeypot logs
\end{Verbatim}

\section{Cron}
\begin{Verbatim}[commandchars=\\\{\}]
Auto update
auto remove unused packages
logrotate
\end{Verbatim}

\section{Honepot}

\section{SSH}
\subsection{Installation}
\begin{Verbatim}[commandchars=\\\{\}]
openssh-client
openssh-server
\end{Verbatim}

\subsection{SSHD Config}
\begin{Verbatim}[commandchars=\\\{\}]
Config (\textcolor{file}{/etc/ssh/sshd_config})(0644):
\textcolor{comment}{# Port:}
Port 22
\textcolor{comment}{# Both: any; IPv4: inet; IPv6: inte6:}
AddressFamily inet
\textcolor{comment}{# Listen Address (0.0.0.0 = any):}
ListenAddress 0.0.0.0

\textcolor{comment}{# Root logins [prohibit-password]:}
PermitRootLogin prohibit-password
\textcolor{comment}{# Allow specific users (separated by spaces):}
AllowUsers pi
\textcolor{comment}{# If ~/.ssh/rc file is executed. [yes]:}
PermitUserRC no

\textcolor{comment}{# Server disconnects, if user did not logged in seconds [120]:}
LoginGraceTime 60
\textcolor{comment}{# Max number of auth attempts.}
\textcolor{comment}{# If number of failures reaches half, additional failures are logged [6]:}
MaxAuthTries 1
\textcolor{comment}{# Max sessions. [10]:}
MaxSessions 1

\textcolor{comment}{# Banner <none> o <PATH/TO/FILE>:}
Banner none
\textcolor{comment}{# If "/etc/motd" is printed [yes]:}
PrintMotd no
\textcolor{comment}{# Logging (QUIET, FATAL, ERROR, INFO, VERBOSE, DEBUG1, DEBUG2, and DEBUG3):}
LogLevel: VERBOSE
\end{Verbatim}
\subsection{PAM Config}
\begin{Verbatim}[commandchars=\\\{\}]
Lock Account (\textcolor{file}{/etc/pam.d/common-auth})(0644):
\textcolor{comment}{After 3 fails lock acc+root for 60s and log in audit:}
\textcolor{comment}{# Add before first config block}
auth required pam_tally2.so deny=3 unlock_time=60
...
\end{Verbatim}
\subsection{Run on startup}
\begin{Verbatim}[commandchars=\\\{\}]
    \textcolor{root}{pi#} \textcolor{command}{systemctl} enable ssh.service
\end{Verbatim}

\subsection{IDS}
\begin{Verbatim}[commandchars=\\\{\}]
\url{https://www.howtogeek.com/675010/how-to-secure-your-linux-computer-with-fail2ban/}
\url{http://support.moonpoint.com/os/unix/linux/centos/fail2ban-logging.php}
\end{Verbatim}


\subsection{ToDo}
\begin{Verbatim}[commandchars=\\\{\}]
ssh key pair + password
logging
error message too many auth fails

Auto software update - cron
IPS - fail2ban
FW logs
\end{Verbatim}


\section{VNC}
\subsection{VNC server installation}
\begin{Verbatim}[commandchars=\\\{\}]
SERVER Install x11vnc server:
    \textcolor{root}{pi#} \textcolor{command}{apt} install x11vnc

CLIENT VNC software:
    \textcolor{root}{root#} \textcolor{command}{apt} install [xtightvncviewer|tigervnc-viewer]
\end{Verbatim}

\subsection{VNC server test}
\begin{Verbatim}[commandchars=\\\{\}]
Test-run x11vnc server (UNSECURE, NO PASSWORD!)
SERVER:
    \textcolor{user}{pi\$} \textcolor{command}{x11vnc} -display :0

CLIENT:
    \textcolor{user}{client\$} \textcolor{command}{vncviewer} <IP_ADDRESS>::<PORT>
\end{Verbatim}

\subsection{VNC server Run and Connect to}
\begin{Verbatim}[commandchars=\\\{\}]
Run server with SSH tunnel:
SERVER (run 512MB for GFX memory):
    \textcolor{user}{pi\$} \textcolor{command}{x11vnc} -display :0 -rfbport 5900 -no6 -input "KMBCF" -forever 
    -nevershared -allow "127.0.0.1" -localhost -listen "localhost" 
    -noxdamage -logappend \textcolor{file}{~/.vnc/logs} [-ncache 10] -usepw

CLIENT:
Window1:
    \textcolor{user}{user\$} \textcolor{command}{ssh} -t -L 11111:localhost:5900 pi@<IPv4>
Window2:
    \textcolor{user}{user\$} \textcolor{command}{vncviewer} localhost:11111
\end{Verbatim}

\subsection{VNC change password during live run}
\begin{Verbatim}[commandchars=\\\{\}]
\textcolor{user}{pi\$} \textcolor{command}{vncpasswd}

Password file is stored in:
\textcolor{file}{~/.vnc/passwd}
\textit{Only first 8 chars are required to login.}
\end{Verbatim}

\subsection{VNC after reboot}
\begin{Verbatim}[commandchars=\\\{\}]
NOT YET

Lets Try tigervnc
\end{Verbatim}






\section{ToDo}
\begin{Verbatim}[commandchars=\\\{\}]
VNC start on boot
DNS + DNS Filtering
Static IP
Doas
\end{Verbatim}


\chapter{ToDo}
\section{Check Temperature}
\begin{Verbatim}[commandchars=\\\{\}]
Returns the temperature of the SoC as measured by the on-board temperature sensor

vcgencmd measure_temp

Install SW:
git, htop

SW needed:
vim
git
\end{Verbatim}





\end{document}
